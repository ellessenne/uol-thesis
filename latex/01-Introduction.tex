%!TEX root = YEAR-SURNAME-N-PhD.tex

\chapter{Introduction}\label{intro}

If you are thinking about using the \LaTeX\xspace template, think again.

If you are still convinced after thinking about it, good luck and read what's coming next.
To minimise headache, I recomment using \href{https://www.overleaf.com/}{Overleaf}: with that you don't have to manage your \LaTeX\xspace installation (thanks UoL IT) and it supports version control out of the box (if you're into that kind of thing).

The titlepage can be modified in \texttt{titlepage.tex}; the rest of the frontmatter in \texttt{frontmatter.tex}.
Most customisations can be modified in the \texttt{preamble.tex} file; however, there is a lot more that could be changed.

If you need help, open an issue on \href{https://github.com/ellessenne/uol-thesis}{this GitHub repository} and mention \texttt{latex}.

\section{Section}\label{intro-section}

This thesis template follows the guidelines from the University of Leicester for PhD theses.

This document uses plain \LaTeX, despite:

\begin{quote}
``Plain \LaTeX\xspace is just so unnecessarily complicated, make your life easier and use \texttt{bookdown}!''

-- Alessandro Gasparini
\end{quote}
